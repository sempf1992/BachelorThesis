\documentclass[../Thesis.tex]{subfiles}
\begin{document}
In the previous chapter we shown that 
\begin{equation}
\tilde{K}(S^2X) \rightarrow \tilde{k}(S^2A) \rightarrow \tilde{K} S(X/A) \rightarrow \tilde{K}(SX) \rightarrow \tilde{K}(SA) \rightarrow \tilde{K}(X/A) \rightarrow \tilde{K}(X) \rightarrow \tilde{K}(A)
 \end{equation}
is an exact sequence of groups. By definition of the graded topological K-theory we have the exact same sequence:
\begin{equation}
\tilde{K}^{-2}(X) \rightarrow \tilde{k}^{-2}(A) \rightarrow \tilde{K}^{-1} (X,A) \rightarrow \tilde{K}^{-1}(X) \rightarrow \tilde{K}^{-1}(A) \rightarrow \tilde{K}^0(X,A) \rightarrow \tilde{K}^0(X) \rightarrow \tilde{K}^0(A)
 \end{equation}
Now by the product theorem we have that $\tilde{K}^{-2}(X)$ is isomorphic to $\tilde{K}^0(X)$ by multiplying every element in $\tilde{K}^0(X)$ by $\beta$. This makes it reasonable by setting $\tilde{K}^{2i}(X) = \tilde{K}(X)$ and $\tilde{K}^{2i +1}(X) = \tilde{K}(SX)$. Then we have the following exact sequence of groups:
\begin{center}
 \\\begin{tikzpicture}
  \matrix (m) [matrix of math nodes,row sep=3em,column sep=2em,minimum width=2em]
  {
\  \tilde{K}^0(X,A)& \tilde{K}^0(X) & \tilde{K}^0(A) \\
   \tilde{K}^1(X,A)& \tilde{K}^1(X) & \tilde{K}^1(A) \\
   };
  \path[-stealth]
    (m-1-1) edge node [above] { } (m-1-2)
    (m-1-2) edge node [above] { } (m-1-3)
    
    (m-1-1) edge node [right] {} (m-2-1)

    (m-2-1) edge node [below] {} (m-2-2)
    (m-2-2) edge node [below] {} (m-2-3)
    
    (m-2-3) edge node [right] {} (m-1-3);

\end{tikzpicture}    
\end{center}
Now we can define a product between $\tilde{K}^{i}(X) \otimes \tilde{K}^{j}(Y) \rightarrow \tilde{K}^{i + j}(X \wedge Y)$. We have an external product on $\tilde{K}(X) \otimes \tilde{K}(Y)$ which goes to $\tilde{K}(X \wedge Y)$. Then if we replace $X$ by $S^iX$ and $Y$ by $S^jY$ we have our product. By our above statement  we can define $\tilde{K}^*(X)$ to be $\tilde{K}^0(X) \oplus \tilde{K}^{-1}(X)$ without losing any information. Hence we have a product of the form
$$\tilde{K}^*(X) \otimes \tilde{K}^*(Y) \rightarrow \tilde{K}^*(X \wedge Y)$$
We can also define a relative form of this product:
$$\tilde{K}^*(X,A) \otimes \tilde{K}^*(Y,B) \rightarrow \tilde{K}^*(X \times Y, X \times B \cup A \times Y)$$
We define this product by replacing $\tilde{K}^i(X/A)$ by $\tilde{K}(\Sigma^i(X/A)$. Since $X/A \wedge Y/B = (X \times Y) / (X \times B \cup A \times Y)$ we can replace this in the definition of $\tilde{K}^{i + j}(X/A \wedge Y/B)$ and get $\tilde{K}^{i + j} (X \times Y, X \times B \cup A \times Y)$
\\We can now compose the product $\tilde{K}^*(X) \otimes \tilde{K}^*(X) \rightarrow \tilde{K}^*(X \wedge X)$ with the map $\tilde{K}^*(X \wedge X) \tilde{K}^*(X)$ which comes from the map $x \mapsto (x,x)$. This gives a multiplication on $\tilde{K}^*(X)$. This makes $\tilde{K}(X)^*$ into a ring, which extends the ring on $\tilde{K}^0(X)$. Again notice that there need not be an identity, and all other axioms are satisfied.
\\We can also give a relative form of this product in the same way as before, $\tilde{K}^*(X,a) \otimes \tilde{K}^*(X,B) \rightarrow \tilde{K}^*(X, A \cup B)$ which comes from the diagonal map $X/(A \cup B) \rightarrow X/A \wedge X/B$. 
Prop 2.14
Prop 2.15
\end{document}
