\documentclass[../Thesis.tex]{subfiles}
\begin{document}
In the previous chapter we shown that 
\begin{equation}
\tilde{K}(S^2X) \rightarrow \tilde{k}(S^2A) \rightarrow \tilde{K} S(X/A) \rightarrow \tilde{K}(SX) \rightarrow \tilde{K}(SA) \rightarrow \tilde{K}(X/A) \rightarrow \tilde{K}(X) \rightarrow \tilde{K}(A)
 \end{equation}
is an exact sequence of groups. By definition of the graded topological K-theory we have the exact same sequence:
\begin{equation}
\tilde{K}^{-2}(X) \rightarrow \tilde{k}^{-2}(A) \rightarrow \tilde{K}^{-1} (X,A) \rightarrow \tilde{K}^{-1}(X) \rightarrow \tilde{K}^{-1}(A) \rightarrow \tilde{K}^0(X,A) \rightarrow \tilde{K}^0(X) \rightarrow \tilde{K}^0(A)
 \end{equation}
Now by the product theorem we have that $\tilde{K}^{-2}(X)$ is isomorphic to $\tilde{K}^0(X)$ by multiplying every element in $\tilde{K}^0(X)$ by $\beta$. This makes it reasonable by setting $\tilde{K}^{2i}(X) = \tilde{K}(X)$ and $\tilde{K}^{2i +1}(X) = \tilde{K}(SX)$. Then we have the following exact sequence of groups:
\begin{center}
 \begin{tikzpicture}
  \matrix (m) [matrix of math nodes,row sep=3em,column sep=2em,minimum width=2em]
  {
\  \tilde{K}^0(X,A)& \tilde{K}^0(X) & \tilde{K}^0(A) \\
   \tilde{K}^1(X,A)& \tilde{K}^1(X) & \tilde{K}^1(A) \\
   };
  \path[-stealth]
    (m-1-1) edge node [above] { } (m-1-2)
    (m-1-2) edge node [above] { } (m-1-3)
    
    (m-1-1) edge node [right] {} (m-2-1)

    (m-2-1) edge node [below] {} (m-2-2)
    (m-2-2) edge node [below] {} (m-2-3)
    
    (m-2-3) edge node [right] {} (m-1-3);

\end{tikzpicture}    
\end{center}
Now we can define a product between $\tilde{K}^{i}(X) \otimes \tilde{K}^{j}(Y) \rightarrow \tilde{K}^{i + j}(X \wedge Y)$. We have an external product on $\tilde{K}(X) \otimes \tilde{K}(Y)$ which goes to $\tilde{K}(X \wedge Y)$. Then if we replace $X$ by $S^iX$ and $Y$ by $S^jY$ we have our product. By our above statement  we can define $\tilde{K}^*(X)$ to be $\tilde{K}^0(X) \oplus \tilde{K}^{-1}(X)$ without losing any information. Hence we have a product of the form
$$\tilde{K}^*(X) \otimes \tilde{K}^*(Y) \rightarrow \tilde{K}^*(X \wedge Y)$$
We can also define a relative form of this product:
$$\tilde{K}^*(X,A) \otimes \tilde{K}^*(Y,B) \rightarrow \tilde{K}^*(X \times Y, X \times B \cup A \times Y)$$
We define this product by replacing $\tilde{K}^i(X/A)$ by $\tilde{K}(\Sigma^i(X/A)$. Since $X/A \wedge Y/B = (X \times Y) / (X \times B \cup A \times Y)$ we can replace this in the definition of $\tilde{K}^{i + j}(X/A \wedge Y/B)$ and get $\tilde{K}^{i + j} (X \times Y, X \times B \cup A \times Y)$
\\We can now compose the product $\tilde{K}^*(X) \otimes \tilde{K}^*(X) \rightarrow \tilde{K}^*(X \wedge X)$ with the map $\tilde{K}^*(X \wedge X) \tilde{K}^*(X)$ which comes from the map $x \mapsto (x,x)$. This gives a multiplication on $\tilde{K}^*(X)$. This makes $\tilde{K}(X)^*$ into a ring, which extends the ring on $\tilde{K}^0(X)$. Again notice that there need not be an identity, and all other axioms are satisfied.
\\We can also give a relative form of this product in the same way as before, $\tilde{K}^*(X,a) \otimes \tilde{K}^*(X,B) \rightarrow \tilde{K}^*(X, A \cup B)$ which comes from the diagonal map $X/(A \cup B) \rightarrow X/A \wedge X/B$. 
\begin{prop}
Multiplication in $\tilde{K}^*(X)$ is commutative up to a sign: $\alpha \beta = (-1)^{ij} \beta \alpha$ for all $\alpha \in \tilde{K}^i(X)$ and $\beta \in \tilde{K}^j(X)$.
\end{prop}
\begin{myproof}
The product is as we have just seen the composition
\begin{equation}
\tilde{K}(S^i \wedge X) \otimes \tilde{K}(S^j \wedge X) \rightarrow \tilde{K}(S^i\wedge S^j \wedge X \wedge X) \rightarrow \tilde{K}(S^i \wedge S^j \wedge X)
\end{equation}
where the first map is the external product and the second map is induced by the diagonal map. If we replace $\alpha \beta$ by $\beta \alpha$ the factors in the first term $\tilde{K}(S^i \wedge X) \otimes \tilde{K}(S^j \wedge X)$ switch. This corresponds to switching the $S^i$ and the $S^j$ factors in the second term, and also in the third therm. If we view $S^i \wedge S^j$ as $i+j$ wedges of $S^1$, then switching $S^i$ and $S^j$  is the same as $ij$ times switching two adjacent circles. So all we need to show is that if we switch two adjacent factors we get a $-1$ sign and then we have proven that $\alpha\beta = (-1)^{ij} \beta \alpha$.
\\If we transpose the two factors of $S^1 \wedge S^1$. it is equivalent to reflecting $S^2$ across an equator, since the smash product of two $S^1$ is homotopic to $S^2$, and if we switch them the homotopy reflects $S^2$ across its equator. Since reflecting across the equator of $S^2$ is the same as reversing the direction in which we do a suspension, since $S^2 = SS^1$. Hence all we need to do is show that reversing the two ends of a suspension $SY$ induces multiplication by $-1$ in $\tilde{K}(SY)$.
We can view $\tilde{K}(SY)$ as $[Y,U]$, where $U$ is the infinite unitary group, we see that switching ends of $SY$ corresponds to the map $U \rightarrow U$ sending a matrix to its inverse. The group operation induced by this map is the same as the operation induced by the product in $U$. Hence we see that reversing the end points of a suspension gives rise to a factor $-1$
\end{myproof}
We will state the following result without prove for completeness:
\begin{prop}
The following sequence is an exact sequence of $\tilde{K}^*$ modules with maps homomorphisms of $\tilde{K}^*$ modules:
\begin{center}
 \begin{tikzpicture}
  \matrix (m) [matrix of math nodes,row sep=3em,column sep=2em,minimum width=2em]
  {
\  \tilde{K}^*(X,A)& \tilde{K}^0(X)\\
   \tilde{K}^*(A)\\
   };
  \path[-stealth]
    (m-1-1) edge node [above] { } (m-1-2)
    (m-1-2) edge node [above] { } (m-2-1)
    (m-2-1) edge node [above] { } (m-1-1);

\end{tikzpicture}    
\end{center}
\end{prop}
We will now show how to create unreduced versions of the group $\tilde{K}(X)$. We define $K^n(X)$ to be $\tilde{K}^n(X_+$, where $X$ is just $X$ with a disjoint extra point called $+$ adjoined. For $n = 0$ this is consistent with the established relations between $\tilde{K}(X)$ and $K(X)$ since $K^0(X) = \tilde{K}^0(X_+) = \tilde{K}(X_+) = \ker( K(X_+) \rightarrow K(+)) = K(X)$. When $n = 1$ this yields $K^1(X) = \tilde{K}^1(X)$ since $S(X_+) = SX \vee S^1$ and $\tilde{K}(SX \vee S^1) \approx \tilde{K}(SX) \oplus \tilde{K}(S^1) = \tilde{K}(SX)$. Here we used $\tilde{K}(S^1) = 0$. For a pair $(X,A)$ with $A$ nonempty we define $K^n(X,A) = \tilde{K}^n(X,A)$ and the six term exact sequence which we derived at the beginning of this chapter is again valid. If $A$ is empty then the statement remains true if we consider $X/\emptyset = X_+$.
\\We now work on creating an product, the idea is again the same. Since $X_+ \wedge Y_+ = (X \times Y)_+$ the external product $\tilde{K}^*(X) \otimes \tilde{K}^ *(Y) \rightarrow \tilde{K}^* ( X \wedge Y)$ gives a product $K^*(X) \otimes K^*(Y) \rightarrow K^*(X \times Y)$. If we then again take $ Y = X$ and compose with the diagonal map we get a product which makes $K^*(X)$ into a ring.
\\We can do the same for the relative product $K^i(X,A) \otimes K^j(Y,B) \rightarrow K^{i + j} ( X \times Y, X \times B \cup A \times Y)$ defined as the external product $\tilde{K}(\Sigma^i(X/A)) \otimes \tilde{K}(\Sigma^j(Y/B)) \rightarrow \tilde{K}( \Sigma^{i + j}(X/A \wedge Y/B))$ using the same identification as before: $(X \times Y)/(X \times B \cup A \times Y) = X/A \wedge Y/B$. Again this works when $A = \emptyset$ since we interpret $X/\emptyset$ as $X_+$ and similarly for $(Y,B)$. Via the diagonal map we also obtain a product $K^i(X,A) \otimes K^j(X,B) \rightarrow K^{i + j} (X, A \cup B)$. With these definitions the preceding propositions also become true for $K$ groups.
\end{document}
