\documentclass[../Thesis.tex]{subfiles}
\begin{document}
In topological K-theory we develop an Functor $K$ which maps compact Hausdorff topological spaces to rings. So given a topological space $X$ we are going to develop a ring $K(X)$. A ring has to obey the following axioms:
\begin{itemize}
\item $K(X)$ is an abelian group under addition.
\item $K(X)$ is a monoid under multiplication, eg $\cdot$ is associative and there is an identity.
\item multiplication is distributive with respect to addition: $a\cdot ( b+c) = (a\cdot b) + (a \cdot c)$ and $(b+c)\cdot a = (b\cdot a) + (c \cdot a)$
\end{itemize}




So first we need a abelian group for addition. After that, we need an multiplication operation which distributes over the addition. We shall show that the direct sum of vector bundles on $X$ can be made into an abelian group, and that tensoring vector bundles over $X$ defines a multiplication.
\\Furthermore, to be a functor, morphisms from one topological space into the other should induce morphisms from one ring to the other. In this case, the morphisms from one topological space to the other is a continuous function, and the morphisms from the one ring to the other are ring homomorphisms. In other words, the following diagram commutes:\\\newline

\begin{tikzpicture}
  \matrix (m) [matrix of math nodes,row sep=3em,column sep=4em,minimum width=2em]
  {
     X & Y \\
     K(X) & K(Y) \\};
  \path[-stealth]
    (m-1-1) edge node [above] {$F$} (m-1-2)
    (m-1-1) edge node [left] {$K$} (m-2-1)
    (m-2-1) edge node [below] {$F_*$} (m-2-2)
    (m-1-2) edge node [right] {$K$} (m-2-2);
\end{tikzpicture}    
\section{Direct sum of vector bundles induces an abelian group}
We start with a few basic constructions. First we fix a compact Hausdorff space $X$. It is convenient to take a slightly broader definition of vector bundles in this chapter. We allow fibers of a vector bundle $p : E \rightarrow X$ to be vector spaces of different dimensions. There still should be local trivializations $h: p^{-1}(U) \rightarrow U \times \mathbb{C}^n$. 
\begin{lemma}
Dimensions of fibers of vector bundles are locally constant.
\end{lemma}
\begin{myproof}
Suppose that we have an open cover $\mathcal{U} = \{U_\alpha\}$ of $X$. Then we can split every open $U_\alpha$ in $\mathcal{U}$ into connected components $V_{\alpha, \beta}$. This gives another open cover $\mathcal{V}$.
\\Then note that for any $V_{\alpha, \beta}$ the space $V_{\alpha, \beta} \times \mathbb{K}^n$ has constant dimension, due to dimension being locally constant in $V_{\alpha, \beta}$ and $\mathbb{K}^n$ having dimension $n$.
\\Now suppose that we have two opens $V_1,V_2$ such that $p^{-1}(V_1) \cap p^{-1} (V_2) = W$ is non empty. Then we have two homeomorphisms $p : W \rightarrow p(W) \subset V_1$ and $p : W \rightarrow p(W) \subset V_2$. Hence $W$ has the same dimension as $V_1$ and $V_2$. Thus they are equal. This shows that dimension is locally constant.
\end{myproof}
We now define a equivalence relation on the set of all vector bundles. 
\begin{mydef}
We call two vector bundles $E_1$ and $E_2$ equivalent, $E_1 \sim E_2$ iff there exist trivial vector bundles $\epsilon^n, \epsilon^m$ of dimension $n$ and $m$ such that $E_1 \oplus \epsilon^n \approx E_2 \oplus \epsilon^m$
\end{mydef}
We shall show that the equivalence classes of $\sim$ form an abelian group.
\begin{theorem}
The equivalence classes of $\sim$ form an abelian group with the operation $\oplus$.
\end{theorem}
We define $[E_1] \oplus [E_2] = [E_1 \oplus E_2]$. This is clearly closed since any direct sum of vector bundles is again a vector bundle. It is also associative and the class of the trivial bundle $[\epsilon^0]$ forms the identity $[E_1] + [\epsilon^0] = [E_1 \oplus \epsilon^0] = [E_1]$. Thus the only thing we need to show is that there exist inverses.
\\We have shown that if for a vector bundle $E$ all the fibers have the same dimension $n$, there exist an vector bundle $E'$ such that their direct sum is a trivial vector bundle.
\\Now suppose that we have a vector bundle $p: E \rightarrow X$. We define $X_i = \{ x \in X| dim p^{-1} (x) = i\}$. This is a disjoint set in $X$, is open and hence forms an open cover. By compactness we can extract a finite subcover, but we cannot leave any non open cover out due to every open being disjoint from the rest. Hence there are only a finite amount of such $X_i$. We can find for each such $X_i$ a vector bundle $E'_i$ such that the direct sum of $p^{-1} X_i \oplus E'_i$ is trivial. These $E'_i$ together can then be made the fibers of a vector bundle over $X$ and we have found our inverse. This forms an abelian group $\tilde{K}(X)$. We shall construct $K(X)$ by using a stronger equivalence relation $\approx_s$.
\begin{mydef}
We call two vector bundles $E_1, E_2$ over $X$ stably isomorphic, $E_1 \approx_s E_2$ if $E_1 \oplus \epsilon^n \approx E_2 \oplus \epsilon ^n$
\end{mydef}
Note that if $E_1 \approx_s E_2$ then $E_1 \sim E_2$. Now we construct the abelian group $K(X)$ using $\approx_s$. We cannot have inverses in the same way as in $\tilde{K}(X)$ as two positive vector bundles cannot be added to form the zero dimensional trivial vector bundle. Thus we seek another property. Observe the following: 
\begin{lemma}
If $E_1 \oplus E_2 \approx_s E_1 \oplus E_3$ then $E_2 \approx_s E_3$.
\end{lemma}
\begin{myproof}
We know that $E_1 \oplus E_2 \approx_s E_1 \oplus E_3$ and we know that for $E_1$ there exists an $E'_1$ such that $E_1 \oplus E'_1 \approx \epsilon^n$ for some $n$. Now add this $E'_1$ to both sides of the first equation. This yields
$E'_1 \oplus E_1 \oplus E_2 \approx \epsilon^n \oplus E_2 \approx_s E_3 \oplus \epsilon ^n$. And thus $E_2 \approx_s E_3$.
\end{myproof}
We will use formal differences of vector bundles to form an abelian group. Notice that the zero element is equivalent to $E-E$ for any vector bundle $E$. 
\begin{theorem}
$K(X) = \tilde{K}(X) \oplus \mathbb{Z}$
\end{theorem}
\begin{myproof}
Since $\approx_s$ was a stronger equivalence relation, we can find a natural homomorphism $K(X) \rightarrow \tilde{K}(X)$. Notice that the homomorphism is surjective, and the kernel consist of elements of the form $\epsilon^m - \epsilon^n$. This subgroup of $K(X)$ is isomorphic to $\mathbb{Z}$. 
\end{myproof}
\section{Tensoring vector bundles induces a multiplication}
We have just defined the additive structure of $K(X)$ and are going to work on the multiplicative structure of $K(X)$. If we have two elements of $K(X)$, namely $E_1, E_2$, we define their product as follows $E_1 \otimes E_2$. Since the elements of $K(X)$ are in general represented by differences of vector bundles $E - E'$ we can define their product as follows:

\begin{equation} 
(E_1 - E'_1)(E_2 - E'_2) = E_1 \otimes E_2 - E_1 \otimes E'_2 - E'_1 \otimes E_2 + E'_1 \otimes E'_2
\end{equation}

Notice that tensoring is associative. distributively follows from the definition. All we have to show is that there exists an identity. $\epsilon^1$ is the identity by $\epsilon^1 \otimes ( E - E') = \epsilon^1 \otimes E - \epsilon \otimes E'= E - E'$. By commutativity $\epsilon^1$ is the multiplicative identity. Also notice that $\epsilon^n \otimes E$ gives $n$ copies of $E$ hence we can abbreviate $\epsilon^n$ by $n$ and let $nE$ stand of $n$ copies of $E$.
\\Now we want to show that this multiplication is well defined. Suppose $E$ and $E'$ are in the same equivalence class, and we multiply this by $E_2 - E'_2$. 







As we will see in the next section, continuous maps induce ring homomorphisms. If we take the map which maps $X$ to $x_0 \in X$ this comes down to restricting vector bundles over the fibers to $x_0$. Its kernel is $\tilde{K}(X)$. Hence it is an ideal. Then $\tilde{K}(X)$ obeys all the ring axioms except it might not have an identity element.
\section{Continuous maps induce ring homomorphisms}
Given two compact Hausdorff spaces $X, Y$ and a continuous map between them. As we have seen, a map $f: X \rightarrow Y$ induces a map in vector bundles $f^*: E_Y \rightarrow E_X$. We have seen that $f^* (E_1 \oplus E_2) = f^*(E_1) \oplus f^*(E_2)$ and $f^*(E_1 \otimes E_2) = f^*(E_1) \otimes f^*(E_2)$ hence $f^*$ is a homomorphism. Further more $(fg)^* = g^* f^*$, $1^* = 1$. Thus $K$ obeys all the required relations for a functor. 

\section{Extra properties of $K(X)$}
These properties will be proved in later chapters
\subsection{External product}
We define the external product $\mu : K(X) \otimes K(Y) \rightarrow K(X \times Y)$ to be $\mu( a \otimes b) = p_1^*(a) p_2^*(b)$ where $p_1$ and $p_2$ are the projects of $X \times Y$ onto $X$ and $Y$. Notice that we tensor rings. The product of $a \otimes b \cdot c \otimes d$ is defined to be $ a \cdot c \otimes b \cdot d$.
\subsection{The fundamental Product theorem}
Theorem: The homomorphism $\mu:K(X) \otimes \mathbb{Z}[H]/(H-1)^2 \rightarrow K(X \times S^2)$ is an isomorphism of rings for all compact Hausdorff $X$.
\\We will not proof this theorem here, as it is a detour from our goal. We might prove it later.
\\This theorem yields directly the following corollary by taking $X = \{0\}$:
\\$\mathbb{Z}[H]/(H-1)^2 \rightarrow K(S^2)$ is an isomorphism of rings. This implies directly that $K(S^2) = \mathbb{Z}$
\subsection{Complex Bott periodicity}
The Bott periodicity theorem for complex vector bundles states that $\beta: \tilde{K}(X) \rightarrow \tilde{K}(S^2 X)$ with $\beta(a) = (H-1) * a$ is an isomorphism for all compact Hausdorff $X$.
\\As a corollary we have that $\tilde{K}(S^{2n + 1}) = 0$ and $\tilde{K}(s^{2n}) = \mathbb{Z}$.
\subsection{K-theory as a Cohomology}
We can make K-theory into a cohomology theory, by defining $\tilde{K}^{-n} (X) = \tilde{K}(S^n X)$. This obeys the axioms of cohomology. This will be shown in the chapter Extending to a cohomology theory. We then set $K^{-n}(X) = K(\Sigma^n X)$, where $\Sigma$ is the reduced suspension, i.e. where we send the north and south pole to the same point.
\subsection{The Splitting Principle}
Given a vector bundle $E \rightarrow X$ with $X$ compact Hausdorff space, there is a compact Hausdorff space $F(E)$ and a map $: F(E) \rightarrow X$ such that induced map $p^* : K^*(X)  \rightarrow K^*(F(E))$ is injective and $P^*(E)$ splits as a sum of line bundles.
\end{document}