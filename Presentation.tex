\documentclass{beamer}
\title{Which spheres are groups and the classification of the real divison algebras}
\author{Stefan Franssen\\4035844}
\begin{document}
\begin{frame}
\maketitle
\end{frame}

\begin{frame}
An algebra $A$ over a field $F$ is:
\begin{itemize}
\item A vector field over $F$
\pause
\item with an multiplication $\cdot$
\pause
\item which is bilinear
\end{itemize}
\pause
\begin{theorem}
$$x(y + z) = xy + xz$$
$$(y + z)x = yx + zx$$
\end{theorem}
\pause
\begin{theorem}
An associative algebra is a ring.
\end{theorem}
\end{frame}

\begin{frame}
A division algebra $A$ over a field $F$ is:
\begin{itemize}
\item A vector field over $F$
\item with an multiplication $\cdot$
\item which is bilinear
\pause
\item with an unit $e$
\pause
\item every nonzero element $x$ has an inverse
\end{itemize}
\pause
\begin{theorem}
An associative division algebra is a field.
\end{theorem}
\end{frame}

\begin{frame}
As a first classification we have Frobenius theorem
\begin{theorem}
The only real associative division algebras $A$ are isomorphic to $\mathbb{R}, \mathbb{C},\mathbb{H}$
\end{theorem}
\end{frame}

\begin{frame}
A proof outline:
We have $\mathbb{R} \in A$
\pause
\\We can split the elements in $A$ in the vector space $\mathbb{R} \times V$.
\pause
\\We find an minimal set $E$ containing all the generators for $V$.
\pause
\\If $E$ is empty, we have something isomorphic to $\mathbb{R}$.
\pause
\\If $E$ contains one element, we have something isomorphic to $\mathbb{C}$
\pause
\\If $E$ contains two elements, we have something isomorphic to $\mathbb{H}$
\pause
\\If $E$ contains more than $2$ elements, use associativity to show that we can express the last element as a linear combination of the other elements, hence can find a smaller $E$.
\end{frame}
\begin{frame}
Composition algebras are algebras together with a quadratic form.
\pause
\begin{theorem}
The only composition algebras over $\mathbb{R}$ are isomorphic to $\mathbb{R}, \mathbb{C}, \mathbb{H}, \mathbb{O}$.
\end{theorem}
\end{frame}

\begin{frame}
Another proof outline:
\pause
\\There is an identity element $e$.
\pause
\\Every nonzero element has an inverse.
\pause
\\We can classify all algebras by doubling.
\pause
\\We can only double $3$ times.
\end{frame}

\begin{frame}
Note that we can make constructions which are like complex numbers, quaternions or octonions for any field $F$. The previous theorem generalises to
\begin{theorem}
The only composition algebras over a field with characteristic not equal to $2$ are isomorphic to the constructions in dimension $1,2,4,8$ extensions.
\end{theorem}
\end{frame}

\begin{frame}
Problem: find $z_i$ as linear combinations of $x_j, y_k$ such that the following equality holds for all $x_i, y_i$
$$(\sum_{i = 1}^n x_i^2)(\sum_{i = 1}^n y_i^2) = \sum_{i = 1}^n z_i^2$$
\pause
\begin{theorem}
This is only possible when $n = 1,2,4,8$.
\end{theorem}
\end{frame}

\begin{frame}
\begin{theorem}
one can define an associative division algebra on $\mathbb{R}^n$ if and only if $S^{n-1}$ is a topological group
\end{theorem}
\pause
\begin{theorem}
$S^{n-1}$ is an H-space if $\mathbb{R}^n$ can be made into a division algebra
\end{theorem}
\end{frame}

\begin{frame}

\begin{theorem}
$S^{n-1}$ is an $H$-space if and only if $n = 1,2,4,8$
\end{theorem}

\pause
By the previous theorem we now can find all the possible dimensions on which an division algebra can be defined.

\end{frame}

\begin{frame}
Recall this theorem:
\begin{theorem}
The only real associative division algebras $A$ are isomorphic to $\mathbb{R}, \mathbb{C},\mathbb{H}$
\end{theorem}
\pause
And the next theorem:
\begin{theorem}
one can define an associative division algebra on $\mathbb{R}^n$ if and only if $S^{n-1}$ is a topological group
\end{theorem}
\pause
These imply
\begin{theorem}

$S^{n-1}$ is a topological group if and only if $n = 1,2,4$
\end{theorem}
\end{frame}
\end{document}