\documentclass{report}
\usepackage{amsmath}
\usepackage{amssymb}
\usepackage{amsthm}
\usepackage{tikz}
\usepackage{subfiles}
\usetikzlibrary{arrows}

\newtheorem{theorem}{Theorem}
\newtheorem{prop}{Proposition}
\newtheorem{lemma}{Lemma}
\newtheorem{Cor}{Corollary}
\newtheorem{myproof}{Proof}

\newtheorem{mydef}{Definition}
\title{Which spheres are groups}
\author{Stefan Franssen \\4035844}
\usetikzlibrary{matrix}
\linespread{1.5}


\usepackage{geometry}
 \geometry{
 a4paper,
 total={210mm,297mm},
 left=10mm,
 right=10mm,
 top=10mm,
 bottom=20mm,
 }

\newcommand{\Ima}{\text{Im}}
\newcommand{\inner}[2]{ \langle #1, #2 \rangle}
\begin{document}
\maketitle
\tableofcontents{}
\chapter{Introduction}
\subfile{Introduction}
\chapter{Algebras}
\subfile{Algebras}
\chapter{Classification of all composition algebras}
\subfile{CompositionAlgebras}
\chapter{Topological groups}
\subfile{TopologicalGroups}
\chapter{Lie groups}
\subfile{LieGroups}
\chapter{Vector Bundles}
\subfile{VectorBundles}
\chapter{Topological K-theory}
\subfile{TopologicalKTheory}
\chapter{Complex Bott periodicity}
\subfile{ComplexBottPeriodicity}
\chapter{Extending to a cohomology theory}
\subfile{ExtendingToCohomologyTheory}
\chapter{$S^n$ is an H-space iff $n = 0,1,3,7$}
\subfile{SnHopfSpace}

\iffalse
\chapter{Splitting principle}
prop 2.23
prop 2.24
Theorem 2.25
Example 2.26
Proof of splitting principle
\chapter{ToDoList}
\section{Introduction}
Making a more extensive introduction, show an overview of the thesis and explain what we will do per part.
\section{Algebras}
Show the proof of the Cayley-Hamilton theorem
\section{Topological groups}
Give some more examples.
\section{Lie groups}
\section{vector bundles}
Depending on what all gets added, add extra theorems which will be needed later.
\section{Topological K-theory}
\section{Extending to a cohomology theory}
\section{$S^n$ is an H-space iff $n = 0,1,3,7$}
Show naturality
\section{complex Bott periodicity}
\section{Splitting principle}
write chapter
\section{Literature list}
Make a literature list
Add Hatcher - vector bundles and K-theory,
Wikipedia proof of frobenius theorem
proof for the composition algebras
proof for Lie groups
Hatcher - Algebraic geometry
Diktaat topology
Adams - classifying vector fields on spheres

\section{General work}
blz 18 goed bekijken (zei je zelf)
\\proof 18 op blz 23 is een lastig geformuleerde zin

\section{stack}
Classify all complex vector bundles by clutching functions
\\fibrations and vector bundles
\\Proof of existence of Adams operations, proof of naturality must be done
\\In bott periodicity theorem, why unique element in $\tilde{K}(X \wedge Y)$
\\Write chapter on the splitting principle.
\\Fix literature list
\\MOAR examples
\fi
\begin{thebibliography}{8}
\bibitem{VBKT}
Allen Hatcher, vector bundles and K-theory, Preprint from website
\bibitem{Frob}
Wikipedia, Frobenius theorem of real division algebras, wikipedia
\bibitem{Hurwitz}
Rasmus Pr\'ecenth, The (1,2,4,8)-Theorem for Composition Algebras, Juni 2013
\bibitem{Lie1}
Planetmath, Spheres that are Lie Groups,
http://planetmath.org/spheresthatareliegroups
\bibitem{Lie2}
Jason DeVito, Answer on A question, https://math.stackexchange.com/questions/12453/is-there-an-easy-way-to-show-which-spheres-can-be-lie-groups,  nov-2010
\bibitem{AG}
Allen Hatcher, Algebraic geometry, ISBN 0-521-79540-0, Cambridge University Press, 2002
\bibitem{topo}
Marius Crainic, Inleiding topology
\bibitem{Adams}
Frank Adams - classifying vector fields on spheres, Annals of Mathematics, 1962
\end{thebibliography}
\end{document}
